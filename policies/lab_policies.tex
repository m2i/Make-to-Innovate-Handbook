% M:2:I Lab Policies

\chapter{Lab Policies}

\section{0620 Lab Space}
The Make to Innovate lab is located in 0620 Howe Hall.  It is shared space that is shared with M:2:I, AeroE Senior Design and on occasion other student organizations.  The 0620 Lab is also on the both the department and the College of Engineering tours.  This means that we often have guests that visit our lab.  The following policies are in place to ensure that everyone enjoys the lab and the lab is maintained and presentable not only to M:2:I groups but to visitors as well.

\section{Lab General Policies}
The lab is open to all students that are currently enrolled in AerE 294X or AerE 494X.  Other courses or organizations that has been approved by either the Aerospace Engineering Department or the M:2:I Program Coordinator may also use the space.  The space is also often on departmental, College of Engineering and other tours and often has guests that may visit the lab.

\subsection{Food and Drink}
Food in the Make to Innovate lab is not allowed at any time. Drinks are acceptable outside the safety glasses required zone and in the conference room.  All alcoholic beverages are off limits in the lab and on school property per ISU policy. Anyone caught having alcoholic beverages in the lab will be turned into the appropriate authorities.  A group may request to host an event in the conference and request to have food there.  These will be approved on a case by case basis.

\subsection{External Tools/Items}
Lab policy prohibits any tools, supplies, or items in general from being brought into the lab. This includes items intended for temporary use. All needed tools and most consumables will be supplied by the lab. If the has run out or does not carry a certain material/tools, an External Supplies Form must first be filled out and approved. This policy is in place to prevent items that are not lab supplies from being misplaced as well as ensuring that items that are being used in the lab are approved items. 

\subsection{In-Kind Donated Items}
The Make to Innovate Lab welcomes In-Kind donated items as these help to better the lab and quality of learning in the lab. In order to prevent confusion as to what is and is not lab property, donors must fill out the In-Kind Donation Form prior to donating any item to the lab.

\section{Lab Usage Policy}
Students that are using the lab must follow all signs, safety protocols, procedures and instructions from M:2:I Staff, Faculty and Lab Monitors.  Care must be taken that all students have taken all required training and understand that the lab must be used in a safe manor.  

\subsection{Trainings}
All students using the M:2:I lab must have completed the following trainings.  Three of the trainings are from Iowa State University's Environment Health \& Safety (EH \& S) website and the last one is on the Make to Innovate Blackboard page.

\subsection{Lab Personnel}
At least one lab monitor will be on duty at all times that the lab is open. Lab monitors will handle all questions that are not related to purchasing, budget or grades. If a student is unable to resolve an issue with a lab monitor, they should contact the M:2:I Program Coordinator to address any concerns.  Any issue involving grading of the course, purchasing or budget questions should go directly to the M:2:I Program Coordinator.  If a student has an issue with a lab monitor or any other M:2:I personnel, they should contact the M:2:I Program Coordinator immediately.  

\subsection{Lab Cleanup Policy}
Upon completion of an individuals work in the lab, the individual will be expected to clean up after themselves. Cleaning includes but is not limited to, cleaning up and returning tools to the tool crib, returning materials to their respective locations, vacuuming off all used equipment, and sweeping within the work area that the individual occupied.
\subsection{Access Policy}
Access to the M:2:I 0620 lab is a privilege and not a right.  To ensure the safety to our students and for the security of the tools, equipment and materials used, access to the lab is closely monitored and is limited to only students that have access to the space.
\section{Regular Hours}
Regular \index{lab hours} are posted near the lab entrance and online. Students are encouraged to make use of the regular lab hours as extended and after hours access is severely limited. The tool crib will only be open for checkout while a lab monitor is on duty and power tools may not be used without a lab monitor present.


\subsection{Extended Hours}
Extended hours may be granted at the discretion of Matthew Nelson. If your team is working in the lab and you know that you would like to stay late, please ask the lab monitor on duty at least an hour before closing. If your team needs short term extended access (weekends or before/after regular hours), fill out the extended access request form at least 3 days in advance and a lab monitor will contact you if they are able to accommodate you.
\subsection{After Hours Access}
Team leads will be granted after hours access to the lab via RFID. Other students may apply for after hours access in writing to Matthew Nelson only if absolutely necessary. No tools can be checked out and no power tools may be used without a lab monitor present.  No student may be in the lab when Howe Hall is closed.  Howe Hall is closed from midnight to 6 a.m. each day.  In addition, Howe Hall may be closed over certain holidays as well.

\section{Tool Policy}
All tools within the 0620 M:2:I lab are monitored and controlled through the M:2:I tool system.  This system allows us to track inventory and to ensure that all teams have equal access to the tools in the lab.  
\subsection{Training}
In order to check out tools, prior training must be completed. Training will be specific to the type of work being completed rather than for a specific tool. For example, foam working will require a single training program that covers the entirety of the foam tools as well as best practices. If the appropriate training has not been completed for the task a student wishes to do, then that student will not be allowed to use the tools required for that task.
\subsection{Consumables}
M:2:I groups may use lab consumables within reason.  Usage will be tracked by the lab monitors and be reported to Matthew Nelson.  Excessive usage may result in teams being charged for usage of these consumables.

Groups from outside M:2:I may not use lab consumables unless they have prior written approval from Matthew Nelson.  At this time M:2:I does not have the ability to charge outside entities.
\subsection{Tool Checkout}
Upon signing up for Make to Innovate and completing basic training, students will be issued an RFID card. This RFID card will connect to an internal database that keeps track of the training each student has completed. Tool checkout is accomplished by swiping their RFID card. If the tool’s training requirements have been met, then a lab monitor will then proceed to check out the tool to the student. It is then the student’s responsibility to ensure that the entirety of the tool is returned in good working condition.
\subsection{Disciplinary Action}
If a tool is returned and does not work or is missing components that were present upon checkout, a disciplinary system will be enacted. The first offense will be a warning and will be marked in the internal database. Additional offenses will be punishable by a fine to be payable to Make to Innovate to replace the tools or components. This fine will be deducted from the team’s budget. The fine will be assessed based off the dollar amount required to replace the tool or component. 
\section{Project Storage}
Each team will be issued a single locker in which they can store their project materials. All lockers are lab property and are subject to all lab policies. All project related materials must be stored in this locker and need to be labeled with the group’s name. The locker is to be locked at all times. No tools or chemicals of any kind are to be kept in group lockers. Tools will be supplied by the lab and can be obtained for use by following the tool checkout procedures. Chemicals that the lab provides or that are purchased by a group must be stored in the flammables locker that is in the tool crib. This is to prevent spills, leaks, and potential fire hazards. 
\subsection{Shelf Space}
In addition to the locker that is assigned to each group, additional shelf space can be obtained for project storage. This shelf space is primarily for storing large items that the group cannot readily fit in their provided locker space. Shelf space will be divided into 2 foot sections and will be allocated to groups who fill out the Additional Space Request form. The amount of space a team receives will be dependent upon the amount of space the group requested, the amount of space available, and the number of other teams that also need space. Space is not guaranteed and is ultimately up to lab monitors and Matthew Nelson to approve.
\subsection{Travelers}
All items are to be stored in either the group’s locker or on shelf space that has been allocated to them. However, if a project needs temporary storage for their project on a work table, they can do so by filling out a traveler form. Travelers are paper forms that must be attached to any item that is to be left outside of a group’s storage areas. These forms allow monitors and other groups know who the project belongs to, how to get a hold of them, and how long the project will be there. Travelers are a temporary form, and are not designed to be used for extended periods of time. The group must fill out all the information on the form and then turn it into a lab monitor. The lab monitor will then either approve or deny the request based on duration of the request, reason for the storage, and upon space that is currently available. Items left out without a traveler will be jailed. In addition to jailing, the first offense will result in a warning with repeated offenses requiring a written apology letter and an escalating grade or monetary penalty.
\chapter{Lab Resources}
0620 Howe Hall has a number of resources that are available to M:2:I students.  Some resources are also shared with other students and/or with the Aerospace Engineering Department.
\section{Computer Room}
The computer room is open to all students both from Make to Innovate as well as outside of the organization. Computer room hours are restricted to lab hours and will not be extended. When the lab closes, all persons using the computer lab will be required to leave regardless of the situation. The lab operates on a first come first serve basis, meaning there is no priority list for use of these computers.
\section{Conference Room}
The lab conference room is a space that has been set aside for use by M2I groups. Groups can set up recurring meeting times by filling out the M2I Conference Room Request form. This form is available online or you can pick up a copy from the lab monitors desk. The form must be turned in at least one day prior to the planned meeting. In order to see what time slots are available, check the Conference Room calendar at http://bit.ly/m2iconference. Forms must be turned into a lab monitor. Lab monitors will then check the available times and if the slot is available, they will add the group to the calendar. Requests are not final until they show up on the calendar.
\section{Laptop Usage}
The Make to Innovate lab has a supply of laptops that can be checked out for school use for both individuals as well as groups. These can be checked out through a lab monitor. In order to check out a laptop, an individual will have to present their Iowa State University ID card, which will be held until the laptop is returned. Once an individual has checked out a laptop, they are then fully responsible for the laptop until they have checked the laptop back into a lab monitor.
\section{Storage Room/Bulk Supplies}
Lab supplies will be stored in the storage room. This room will remained locked at all times and can be accessed via a lab monitor. Lab supplies can be used for a group's project, however each group's consumption will be monitored to avoid excessive use.