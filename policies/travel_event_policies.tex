\chapter{M:2:I Travel Policy}

\section{Introduction}
Make to Innovate supports projects that may need to travel.  Whether this travel is in Iowa, across the United States or abroad, M:2:I wants students to expand their knowledge by participating in activities and events that supports the M:2:I mission.  In order to support projects with their travel needs, M:2:I has established travel policies and procedures for all projects wishing to travel.  

These policies and procedures do not supersede any Aerospace Engineering, College of Engineering or Iowa State University policies and procedures.  Instead, these policies and procedures are meant to complement those procedures and provide students in M:2:I a mechanism to request such travel.  This policy and related procedures are designed to establish the relationship between M:2:I administrative staff and the requesting M:2:I project for the purpose of reducing risks and providing protection for all students participating in the travel.

\section{Policy Statement}
The Make to Innovate Travel Policy and procedures govern travel to reach an activity or event that is authorized by Make to Innovate and is an activity that is required by the project in order to full-fill a project objective, goal or milestone.  All Make to Innovate projects must comply with the requirements for travel as outlined in this policy and any other related procedures.

\section{Authorized Travel}
All travel by recognized M:2:I project must relate to the purpose of the organization and comply with the policies of the State of Iowa; the Board of Regents, State of Iowa; Iowa State University and Make to Innovate.  The ISU policy on Fleet Safety and Vehicle Use/Rental provides information for what is an approved vehicle use.  The purpose of the M:2:I project travel and transportation to and from the event is reviewed and authorized by the project's advisor, the M:2:I Program Coordinator, the Aerospace Engineering Department and the Office of Risk Management prior to travel. The M:2:I project leader must designate a member to serve as the trip coordinator who is responsible for completing trip information on the M:2:I Travel Authorization system.

\section{Eligible M:2:I Projects}
M:2:I projects eligible for travel and use of university vehicles must be in good standing with M:2:I and must not be probation.  Students going on travel must be current students and must be in good standing with Iowa State University.  Travel funding must be secured before requesting travel or the request will be denied.

\section{Forms}
The M:2:I Travel Request form must be completed to request travel for a M:2:I project.  Additional forms may be required dependent on the travel and at the discretion of the M:2:I Program Coordinator and/or Risk Management.

Students under the age of eighteen (18) must have the Waiver and Release of Liability form signed by their parents or legal guardian.

Students using personal vehicles for M:2:I projects travel must sign a Waiver and Release of Liability form and an Emergency Contact and Medical Information form acknowledging the risks involved in the travel activity and assuming responsibility for liability for themselves and any passengers traveling in their vehicle.


\section{Modes of Travel}
M:2:I Project leaders and their designated trip coordinator should consider transportation options and costs before planning a trip. M:2:I Project leaders will be asked to identify their mode of transportation, drivers (if applicable), participant(s) and/or commercial arrangements associated with their trip. Specific policies and procedures as they apply to different modes of transportation are outlined below.

\subsection{Vehicles}
To promote safe driving practices, all M:2:I projects must comply with the requirements for drivers and vehicle use rules outlined in this policy. For the purpose of this policy, vehicles include university vehicles; personal vehicles; vehicles rented, leased or hired by the university; or any vehicle in university control or custody for M:2:I activities.

\subsection{Driver Authorization}

Individuals requesting permission to drive vehicles for authorized M:2:I travel must submit a Motor Vehicle Record Check form to Transportation Services {MVR Check). This form is required to authorize a complete check of the driver's motor vehicle driving record. The individual's motor vehicle record must meet the following minimum qualifications:

\begin{itemize}
\item Driver must be at least 18 years old with the following exception:
\begin{itemize}
\item Drivers of large passenger vans or vehicles towing trailers must be 20 years old and must successfully complete the Transportation Services Large Passenger Van Driving Class.  See Section
\end{itemize}
\item Driver must have a valid U.S. driver's license for the vehicle being driven with the appropriate classifications, restrictions and endorsements.
\item Driver must satisfactorily complete a motor vehicle record check every six months (See Section Driving Standards section below).
\item Driver must agree to operate the vehicle in a safe and prudent manner.
\end{itemize}

Drivers for M:2:I travel must be M:2:I students, or M:2:I Faculty or Staff members.  Student drivers must be members of the project and who are currently enrolled as ISU students and in good standing.  When submitting a request the M:2:I Program Coordinator must used as the contact person.

\subsection{Driving Standards}

Driving privileges for individuals will be denied or revoked if a driver's past twelve-month driving record indicates any of the following:

\begin{itemize}
\item Two citations for a moving violation within the last 12 months.
\item Two accidents within the last 12 months where the driver was at fault or contributory. The definition of "at-fault accident" for this policy means an accident in which the driver is determined to be 50 percent or more responsible for the accident.
\item One accident where the driver was at fault or contributory and one moving violation within the last 12 months.
\item Any citation for blood alcohol content within the last 12 months. Cases not yet resolved in the courts will be considered grounds for temporarily denying permission to drive a university vehicle.
\item A licensing requirement for specialized motor vehicle insurance (i.e., SR) to operate a vehicle.
\item Conviction for reckless driving, driving with a suspended license, hit and run, leaving the scene of an accident, license suspension or other crime(s) that results in license suspension.
\item Conviction or charges pending due to a violation of statutes that affects his/her driver's license, or who has his/her driving privileges suspended, revoked, or barred for violating such statutes including, but not limited to, Operating While Intoxicated, vehicular homicide or habitual violations, or any driving offense punishable as a felony.
\end{itemize}

Individual drivers approved to drive university vehicles for authorized M:2:I travel must notify the M:2:I Program Coordinator and the Office of Risk Management when their driver's license is suspended, revoked, canceled or the driver is otherwise prohibited from operating a university motor vehicle. top

\subsection{Rules and Criteria - All Vehicles}

Passenger Authorization:

\begin{itemize}
\item Authorized passengers include students of the M:2:I project, university employees, or authorized volunteers and are directly involved with the M:2:I program or project.  All passengers must be identified in the M:2:I Travel Authorization system prior to departure.
\item Unauthorized passengers are prohibited in university vehicles. Examples include M:2:I project members not listed on the passenger list, spouses, children or other family members, friends, neighbors or the general public. Unauthorized passengers are not covered by the Regents institutions' insurance.
\end{itemize}

In extenuating circumstances, a request for authorization for passengers otherwise considered unauthorized must be submitted in writing and approved by the Office of Risk Management or Recreation Services (for Sports Clubs) before travel occurs.
Vehicle Occupancy: The maximum number of people in any vehicle must not exceed the number of seatbelts in the vehicle.

In addition the following policies must be observed at all times:

\begin{itemize}
\item All vehicle occupants must wear seat belts at all times while traveling.
\item Transporting people in the bed of a pick-up truck is not allowed on public roads.
\end{itemize}

\subsection{Driving Rules}
The following driving rules must be observed at all times.  Failure to observe these rules may result in revocation of driving privileges.  

The number of drivers required may vary depending on the distance and duration of the trip and must figured in the travel planning.  Drivers are also expected to follow the rules below:

\begin{itemize}
\item Each driver is allowed to drive a maximum of 4 continuous hours followed by a minimum 2-hour break.
\item Each driver is permitted to drive a maximum of 10 hours over a 24-hour period.
\item One person must be in the front passenger seat and awake at all times to assist with navigation and trip safety such as making sure the driver remains alert.
\item Drivers must obey traffic laws and regulations, including posted speed limits.
\end{itemize}

Drivers must abide by university policies and any applicable federal or state regulations that govern individual actions including, but not limited to, ethical behavior, confidentiality, financial responsibility, alcohol and drug use.  No alcoholic beverages or beverage containers (open or closed) are allowed in or around the vehicle. Consumption of alcohol by drivers and passengers is prohibited at all times during the travel and event that the M:2:I project is participating in.  

\subsection{Transporting Hazardous Materials}

The unauthorized transportation, use or storage of any hazardous materials is prohibited. In extenuating circumstances, a request for authorization for transporting hazardous materials must be submitted in writing and approved by the Department of Environmental Health and Safety and the Office of Risk Management before travel occurs. In addition, the Office of Risk Management and the M:2:I Program Coordinator must review and authorize any project activity that involves the use of hazardous materials.

\subsection{Firearms, Weapons and/or Explosives}

The unauthorized transportation, use or storage of any firearms, weapons and/or explosives is prohibited. In extenuating circumstances, a request for authorization for transporting firearms, weapons and/or explosives must be submitted in writing and approved by the Office of Risk Management and the M:2:I Program Coordinator. In addition, the Office of Risk Management and M:2:I Program Coordinator must review and authorize any M:2:I project activity that involves weapon or gun use.

\subsection{Cell Phones and Other Communication Devices}

The use of cell phones and other communication devices such as walkie-talkies while driving is hazardous. Only hands-free units may be used while driving. Drivers are required to stop and park the vehicle to use any other devices.  2-way communication devices such as amateur radio equipment may only be used by the passengers in the vehicle.  Cell phones and other distractions should be kept in a minimum to avoid distractions to the driver.

\subsection{Travel Restrictions}

Travel is not allowed between 1:00 a.m. and 5:00 a.m. on any day.  There are no exceptions to this rule.  In the event of adverse weather or other factors that affect the ability to drive safely, drivers are expected to use good judgment and take appropriate safety measures in observance of travel warnings as issued by the highway safety authorities or weather advisory services.

No items may be transported on the roof of a vehicle.  Rear seats of university vans will not be removed to accommodate luggage without approval from Transportation Services.  Luggage must be dispersed evenly throughout large passenger vans to equalize the load.  When using large passenger vans (12 to 15-passengers) on extended trips, Transportation Services may require a trailer to safely accommodate luggage.

\subsection{Trailers}

Trailers owned by Transportation Services must be used if available.
Transportation Services must approve the use of all commercially rented, privately owned, manufactured, home-made or donated trailers and has the authority to deny the use of any trailer.  Transportation Services must inspect all trailers after connection to the vehicle.  University owned trailers may be pulled only by university owned vehicles.

\section{University Vehicles}\label{ISU_Vehicles}

University vehicles may be used only for official M:2:I travel. All M:2:I projects must comply with the Iowa State University Fleet Safety policy as well as all federal or state regulations that govern related actions including, but not limited to, those of drug and alcohol use, ethical behavior, confidentiality, harassment and financial responsibility. Operating a university vehicle is a privilege. The M:2:I Program Coordinator in conjunction with Transportation Services and/or the Office of Risk Management have the authority to approve or deny any request for the use of university vehicles.

Iowa State University vehicles are easily identifiable. Common sense must be used and consideration must be given to public perceptions of how vehicles are operated and where they are parked. The Iowa Code does not permit personal use of university vehicles and individuals who use vehicles for personal purposes are subject to corrective action or disciplinary measures according to the severity of the infraction and are potentially liable for accidents, injury and damages that occur during unauthorized use.

\subsection{Large Passenger Vans and Vehicles Towing Trailers}

M:2:I projects may be approved to use Iowa State University Transportation Services' 12- and 15-passenger vans for trips with nine to fifteen passengers and/or vehicles towing trailers. Projects may not rent 12- or 15-passenger vans from commercial rental companies or use personal 12- or 15-passenger vans for authorized M:2:I travel.

Due to their unique handling characteristics, drivers of large passenger vans and vehicles towing trailers must be at least 20 years old. In addition, driver training as described below is required.

\subsection{Driver Training}

All drivers of 12- and 15-passenger vans or vehicles towing trailers must complete the Large Passenger Van Driving Class offered by Transportation Services. The Large Passenger Van Driving Class is a two-hour classroom session that covers handling characteristics and defensive driving techniques for 12- and 15-passenger vans.

Each driver must also show behind-the-wheel driving competency by driving a large passenger van with a trailer attached. Competency is determined by the Transportation Services instructor. Behind-the-wheel training will be scheduled after the classroom training is completed.

Each driver must have a record of successful completion of both the classroom and the hands-on, behind-the-wheel training before picking up the keys for a vehicle.  Training records will be kept on file with Transportation Services.

\subsection{Rules and Criteria - University Vehicles}\label{ISU_Vehicle_Criteria}

The following is additional rules and criteria pertaining to ISU University Vehicles.

Smoking is not allowed in Iowa State University vehicles.  All drivers are expected to properly safeguard university vehicles. If it is determined that a vehicle is at substantially higher risk of theft or damage due to a lack of reasonable precautions by the driver or the M:2:I project; the M:2:I project will be notified by the M:2:I Program Coordinator to implement measures to correct the misuse. If the misuse is not corrected within a reasonable time, the M:2:I project may be required to forfeit use of the vehicle and return the vehicle to Transportation Services.  The M:2:I project may also be banned from using ISU vehicles in the future.

All drivers are expected to properly safeguard university vehicles. If it is determined that a vehicle is at substantially higher risk of theft or damage due to a lack of reasonable precautions by the driver; the M:2:I project will be notified by the M:2:I Program Coordinator to implement measures to correct the misuse. If the misuse is not corrected within a reasonable time, the M:2:I project may be required to forfeit use of the vehicle and return the vehicle to Transportation Services.

University vehicles may not be taken into Mexico or Canada without the prior written consent of the Office of Risk Management. Travel into Mexico requires the purchase of Mexican auto insurance and must be arranged through the Office of Risk Management. In addition, the Office of Risk Management must review and authorize any student activity that involves travel to Mexico or Canada.

University vehicles may be driven to a private residence and parked overnight when travel is scheduled to depart early in the morning. Other than short-term travel (i.e. a day or less), private or public transportation should be used to access airports, as it is neither an economical nor effective use of university vehicles to leave them in an airport parking lot.

\section{Personal or Privately Owned Vehicles}

M:2:I projects should minimize the use of personal vehicles for M:2:I-related travel. When a personal vehicle must be used for M:2:I travel, the driver assumes all liability associated with the trip. Drivers and passengers must comply with the M:2:I Travel policy Driving Authorization, Driving Standards and vehicle use Rules and Criteria - All Vehicles. Students using personal vehicles for M:2:I travel must sign a Waiver and Release of Liability form and Emergency Contact and Medical Information form acknowledging the risks involved in the travel activity and assuming responsibility for the liability for themselves and the passengers traveling in their vehicle.

\section{Commercial Travel}
M:2:I projects who use commercial transportation for travel related to M:2:I travel must comply with all university regulations pertaining to commercial travel and the rules of the carrier. This applies to domestic as well as international travel.

\subsection{Air Travel}

Scheduled commercial flights are the most closely regulated and safest form of travel. M:2:I projects who choose other types of flights, such as charters or private planes, must contact the Office of Risk Management regarding contractual agreements and insurance provisions.

\subsection{Chartered Bus or Hired Vehicle}

M:2:I projects may request the use of university contracts for chartered bus or hired vehicle services by making a request to the M:2:I Program Coordinator.

\subsection{Rental Vehicle}

M:2:I projects must rent university Transportation Services vehicles for travel originating in Ames. M:2:I projects starting travel from another location and needing to rent a vehicle commercially must contact the M:2:I Program Coordinator prior to making arrangements.

\section{International Travel}
International travel by M:2:I projects requires extensive planning and preparation. M:2:I projects that wish to travel outside of the United States must complete information in the M:2:I Travel Authorization system and submit it to the M:2:I Program Coordinator at a minimum of six months prior to travel for review and final approval.

\section{Special Circumstances}
Any use of university owned vehicles that involves specific hazards (i.e., HABET Balloon chase, MAVRIC competition, CySLI rocket transportation, etc.) must be reviewed and approved by the M:2:I Program Coordinator and may require additional review from Risk Management.  All requests for exceptions to this policy or any changes to an approved travel itinerary must be submitted to the M:2:I Program Coordinator for approval prior to departure.  The M:2:I Program Coordinator in conjunction with the Aerospace Engineering Department and/or Risk Management will review all special circumstance requests and have the authority to approve or deny any request.

\section{Sanctions}
Failure to comply with any M:2:I Travel policies and procedures may be subject to disciplinary action from the M:2:I Program Coordinator, and/or the Office of Student Conduct disciplinary measures. In addition groups may have travel authorization revoked and may be placed on probation status within M:2:I.  Finally, addition actions may occur that may affect students final grade within the course.

\chapter{Travel Authorization Procedure}

The M:2:I Travel Policy and Procedures for M:2:I projects govern travel for activities or events that are sponsored and authorized by M:2:I.  Travel authorization requests must be submitted and approved prior to travel so the M:2:I staff can properly manage liability issues for student travel.  All forms for travel authorization can be found on the M:2:I Sharepoint site.

\section{Checklist for Travel Authorization}

Checklist for completing your M:2:I Travel Authorization Request:

\begin{itemize}[label={\checkmark}]
\item Click "Add a New Trip" above
\item Search the name of your M:2:I project
\item Complete the "General Information" as completely as possible.
\item Complete all "Travel Information" as completely as possible.
\item Complete all "Lodging Information" as completely as possible.
\item Complete all "Funding Information" as completely as possible.
\item Complete the Itinerary as completely as possible.
\item Complete all "Student Information" as completely as possible.
\item Click "Submit"
\end{itemize}

Once submitted your application will be reviewed by the M:2:I Program Coordinator and any other departments as needed.  Missing information will often delay and may result in your application being denied.  Ensure you have all information provided to help in speeding up the process.  The M:2:I Program Coordinator may request that a special meeting be called in order to get additional information.

In order to use university vehicles, you meet all requirements for using a university vehicle as outlined here and any additional requirements outlined from ISU Transportation.

\subsection{Updating the form}
M:2:I Projects may update their form if needed.  Please note that any update to the form will automatically trigger the need for re-authorization from the M:2:I Program Coordinator.  The list of student going may be updated or changed up to 4 weeks prior to travel.  

M:2:I project travel must relate to the purpose of the project and comply with the policies of the Board of Regents, State of Iowa, and Iowa State University. The purpose of M:2:I project travel and transportation to and from the event will be reviewed and authorized by the M:2:I Program Coordinator and other departments as needed. 

\subsection{Timeline for consideration}

All M:2:I projects that are traveling within the State of Iowa and doing so for an activity or event that will be 24 hours or less must submit an authorization at least 1 week in advance of the time of departure.  If an activity is weather dependent, M:2:I projects should submit a request 1 week in advance and may make changes up to 48 hours in advance of the time of departure.  These requests will typically be reviewed and approved within 48 hours but may take longer if needed or if the form is not completely filled out.

All M:2:I projects that will be traveling outside the State of Iowa for an activity or event that will be 24 hours or less must submit an authorization form at least 2 weeks in advance of the time of departure.  If an activity is weather dependent, M:2:I projects should submit a request 2 weeks in advance and may make changes up to 48 hours in advance of the time of departure.  These requests will typically be reviewed and approved withing 48 hours buy may take longer if needed or if the form is not completely filled out.

All M:2:I projects that will be traveling either within or outside the State of Iowa for an activity or event that will requires any overnight stay must be submitted at least 6 weeks in advance of the time of departure.  Changes to the form may be allowed up to 2 weeks prior to the time of departure.  These requests will typically be reviewed and approved within 1 week.  Additional time may be required depending on the nature of the travel.  This time-line also allows for time to make vehicle and hotel accommodations.

All M:2:I projects that will be traveling outside of the United States for an activity or event must be submitted at least \emph{6 months} in advance of the time of departure.  International travel requires additional permissions from the university which requires extra time.  In addition, additional time is required for securing all necessary paperwork for students that are traveling abroad.  Iowa State University has strict rules on international travel and some locations may not be accessible to studnets for travel.

\section{Trip Coordinator Responsibilities}
The M:2:I Project Leader must designate a member of the travel party to serve as the Trip Coordinator. The Trip Coordinator is responsible for submitting travel authorization requests through the M:2:I Sharepoint Site for each trip. Trip Coordinators must travel with the organization and act as liaison for their M:2:I project during the approval process and on the trip. The Trip Coordinator should confirm receipt of appropriate travel documents with M:2:I faculty and staff. When necessary, the Trip Coordinator may need to meet with staff to review and evaluate the completed M:2:I travel authorization information.

When using a 15-passenger van or a vehicle towing a trailer, the Trip Coordinator must ensure that drivers for the trip have completed Large Passenger Van training. Drivers who need to complete this training must contact Transportation Services and allow for enough time to register for and complete the next available class. These classes are available only through Transportation Services and are usually offered monthly.

\chapter{Event Policy}
Events are any activity that takes place either on or off campus that involves your team.  The only event that is excluded from this is weekly team meetings.  In addition, if the event takes place off campus, even in the Ames, IA area, a travel form must also be filled out.  The following is a list of examples.

\begin{itemize}
\item
Doing an outreach demonstration at a public school
\item
Traveling to a location to do a test flight
\item
Presenting a paper at a conference
\end{itemize}

Events must be pre-approved and if travel is required, that must be approved as well.  Teams are required to give a minimum of 48 hours notice before the event.  Teams are encouraged to provide as much notice as possible.  If the event requires approval from ISU’s Risk Management, there may be a delay in approving the event.  Always consult with the M:2:I Program Coordinator if you have a question on the time needed or if additional requirements are needed.

Some events that a team wishes to do may be dependent on things such as weather that is beyond the control of the team.  In these cases, a team may request am event window.  For these, a window of time will be granted with a maximum window of 1 week.  Teams may then use the best day and time for their event.  If no day or time works within the window, the team will be required to request another event window.